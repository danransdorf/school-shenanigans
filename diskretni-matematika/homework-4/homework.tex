\documentclass[12pt]{article}
\usepackage[a3paper, margin=0.4in]{geometry}
\usepackage{amsmath}
\usepackage{xcolor}
\usepackage{amssymb}


\begin{document}

\section*{Domácí úkol IV}

Vypracoval: Daniel \textit{"Randál"} Ransdorf \hfill Podpis: \rule{4cm}{0.4pt}

\begin{enumerate}
  \item \textbf{Kombinační zmrzlina}

    Vypočítáme, kolik je celkových možností, jak vybrat kopečky. Počet bude velikost množiny možností,
    jak vybrat 8 kopečků ze čtyř příchutí.
    Velikost množiny našeho jevu bude počet možností, jak vybrat 8 kopečků z přesně tří příchutí.
    Pravděpodobnost, že vybereme přesně tři různé příchutě, bude poměr velikosti množiny našeho jevu
    ku velikosti množiny všech možností.

    \begin{enumerate}
      \item Celkové možnosti se dají vypočítat stejně jako příklad ze cvičení: Ze 4 různobarevných hromádek vyber 8 kuliček
        \[
        \binom{n+k-1}{k} = \binom{4+8-1}{8} = \binom{11}{8}
        \]
      \item Počet možností, jak vybrat 8 kopečků přesně ze tří příchutí, se dá interpretovat takto:
        Ze čtyř příchutí si vyberme tři příchutě ($p_1, p_2, p_3$), které použijeme (zjevně 4 možnosti).
        Dejme si 8 kopečků vedle sebe a oddělme je dvěma oddělovači do tří hromádek,
        každá hromádka reprezentuje počet kopečků z dané příchuti.
        \[
          \underbrace{\circ\ \circ\ \circ\ \ \circ}_{p_1}\ |\ \underbrace{\circ\ \circ}_{p_2}\ |\ \underbrace{\circ\ \circ}_{p_3} \\
        \]
        Oddělovače mohu dát mezi kopečky na 7 pozic, nemůžu je dát na kraje,
        protože z každé ze tří příchutí musíme udělat, alespoň jeden kopeček.
        Ilustrace možných pozic oddělovačů:
        \[
        \circ \;|\; \circ \;|\; \circ \;|\; \circ \;|\; \circ \;|\; \circ \;|\; \circ \;|\; \circ
        \]
        Vybírám tedy 2 místa ze 7, kam oddělovače umístím, čili:
        \[ \binom{7}{2} \]
        Započtěme ještě ty čtyři možnosti, kteréma můžeme vybrat tři příchutě ze čtyř, získáme:
        \[ 4 \cdot \binom{7}{2} \]

      \end{enumerate}
    
      Pravděpodobnost získáme poměrem (vysvětlení na začátku):

      \[
        \frac{4 \cdot \binom{7}{2}}{\binom{11}{8}} = \underline{\underline{\ \ \frac{28}{55}\ \ }} \quad \left(\approx 0.509 = {{50.9\%}}\right)
      \]

  \item \textbf{Nezávislost}
    \begin{enumerate}
      \item Pravděpodobnost jevu $A$ je zjevně $\frac{1}{2}$ (26 z 52 karet jsou červené), pravděpodobnost jevu $B$ je $\frac{3}{13}$,
        protože máme 52 karet, z toho 12 jich má obrázek (3 hodnoty $\cdot$ 4 barvy). Pokud by byly jevy nezávislé,
        pravděpodobnost jejich průniku by měla vyjít $\frac{1}{2} \cdot \frac{3}{13} = \frac{3}{26}$.
        Kdyby tomu mělo tak být, vynásobením 52 zjistíme, že by muselo existovat přesně 6 karet,
        jejichž vytažení patří do obou jevů.

        Jev A a zároveň jev B splňují pouze srdcové J, Q, K a kárové J, Q, K --- 6 karet, čimž se nám potvrdilo, že jsou jevy \textbf{nezávislé}.
        Pro kontrolu:
        \[
          P(A \cap B) = \frac{|A \cap B|}{|\Omega|} = \frac{6}{52} = \frac{3}{26} = P(A) \cdot P(B)
        \]
      \item Pravděpodobnost jevu $A$ je $\frac{1}{2}$ (viz 2.a). Když vytáhneme první kartu obrázkovou (pravděpodobnost $\frac{3}{13}$),
        zbývá v balíčku 11 obrázkových karet z 51 karet celkově, pravděpodobnost jevu B by byla tedy $\frac{11}{51}$.
        Když první kartu nevytáhneme obrázkovou (pravděpodobnost $1 - \frac{3}{13} = \frac{10}{13}$),
        v balíčku by zbývalo stále 12 obrázkových karet, pravděpodobnost jevu B by byla $\frac{12}{51}$.
        Sečtením obou možností, když první kartu vytáhneme/nevytáhneme obrázkovou získáme:
        \begin{align*}
          P(B) &= \frac{3}{13} \cdot \frac{11}{51} + \frac{10}{13} \cdot \frac{12}{51} = \frac{3}{13} \\
          P(A) \cdot P(B) &= \frac{1}{2} \cdot \frac{3}{13} = \frac{3}{26}
        \end{align*}

        Průnik obou jevů spočítáme podobně součtem pravděpodobnosti B na základě dvou možností vytažení první karty:
        \begin{enumerate}
          \item Označme $A_{bo}$ jev vytažení červené karty bez obrázku (20 karet z 52): Pravděpodobnost $A_{bo}$ je $\frac{20}{52} = \frac{5}{13}$,
            zbývá 12 karet s obrázkem mezi 51 kartami.
            \[
              P(B \,|\, A_{bo}) = \frac{12}{51}
            \]
          \item Označme $A_{so}$ jev vytažení červené karty s obrázkem (6 karet z 52): Pravděpodobnost $A_{so}$ je $\frac{6}{52} = \frac{3}{26}$,
            zbývá 11 karet s obrázkem mezi 51 kartami.
            \[
              P(B \,|\, A_{so}) = \frac{11}{51}
            \]
        \end{enumerate}
        Započítáním obou možností získáme:
        \begin{align*}
          P(A \cap B) &= P(A_{bo}) \cdot P(B \,|\, A_{bo}) + P(A_{so}) \cdot P(B \,|\, A_{so}) \\
          P(A \cap B) &= \frac{5}{13} \cdot \frac{12}{51} + \frac{3}{26} \cdot \frac {11}{51} \\
          P(A \cap B) &= \frac{3}{26}
        \end{align*}

        Závěrem je, že jevy jsou \textbf{nezávislé}, protože:
        \[
          P(A) \cdot P(B) = \frac{3}{26} = P(A \cap B)
        \]
    \end{enumerate}
\end{enumerate}



\end{document}
