\documentclass[12pt]{article}
\usepackage[a3paper, margin=0.4in]{geometry}
\usepackage{amsmath}
\usepackage{xcolor}
\usepackage{amssymb}


\begin{document}

\section*{Domácí úkol III}

Vypracoval: Daniel \textit{"Randál"} Ransdorf \hfill Podpis: \rule{4cm}{0.4pt}

\begin{enumerate}
  \item {Cukrárna}
    Je k dispozici 11 různých příchutí zmrzliny.

    \begin{enumerate}
      \item Naskládej 3 různé příchutě na sebe $\Longleftrightarrow$ Sestav 3-znakový řetězec z 11-znakové abecedy bez opakování znaků.
          \[
            11^{\underline{3}} = 11 \cdot 10 \cdot 9 = \underline{\underline{990}}
          \]
      \item Zvol 4 kopečky, mohou mít stejnou příchuť $\Longleftrightarrow$ Z 11 různobarevných hromádek vyber 4 kuličky
          \[
            \binom{n+k-1}{k} = \binom{11+4-1}{4} = \binom{14}{4} = \underline{\underline{1001}}
          \]
      \item Sestav dezert z 5 vrstev, příchutě se mohou opakovat $\Longleftrightarrow$ Sestav 5-znakový řetězec z 11-znakové abecedy, můžeš opakovat znaky.
          \[
            11^5 = \underline{\underline{161051}}
          \]
      \item Dezert, kde buď:
        \begin{enumerate}
            \item 5 vrstev tvoří zmrlinové kopečky (mohou se opakovat příchutě) (zdůvodnění viz 1c)
              \[11^5 = 161051\]
            \item nebo 3 vrstvy zmrzlinové a 2 crumble z 5 typů (mohou se opakovat) $\Longleftrightarrow$ Udělej 5-znakový řetězec, kde 3 znaky budou z abecedy A o 11 znakách a 2 znaky budou z abecedy B o 5 znakách. Znaky se mohou opakovat
              \[
                11^3 \cdot 5^2 \cdot \binom{5}{3} = 332750
                \]
                *5 nad 3 přidáme, abychom započetli počet možností, které 3 z 5-ti vrstev mohou být zmrzlinové
        \end{enumerate}
        Případy se navzájem vylučují a pokrývají všechny možnosti, takže celkový výsledek bude součet $161051+332750=\underline{\underline{493801}}$
    \end{enumerate}

  \item \textbf{Rovnost}
    \begin{enumerate}
      \item Tvrzení:
        \[
          \sum_{k=0}^{n} \binom{n}{k}\binom{m}{r-k} = \binom{n+m}{r}
        \]
      \item \textbf{Důkaz 1} - kombinatorická úvaha:

        Mějme množiny $N$ o $n$ prvcích a $M$ o $m$ prvcích, které mají prázdný průnik.
        Chceme vybrat $r$ prvků množin.
        \begin{enumerate}
        
          \item Můžeme to vyjádřit, že vybíráme $r$ prvků ze sjednocení množin $N$ a $M$.
          Počet možností bude:
          \[\binom{n+m}{r}\]
          
          \item Můžeme to také vyjádřit tak, že sečteme všechny možnosti, kolik vybereme prvků z které množiny.
          Takže z množiny $N$ vybereme 0 prvků a z množiny $M$ r prvků,
          pak z množiny $N$ vybereme 1 prvek a z množiny $M$ $r-1$ prvků. A tak dále, až vybereme
          z množiny $N$ všech n prvků a z množiny $M$ 0 prvků.
          Toto se dá zapsat jako suma pro všechny $k \le n$, kde vybíráme k prvků z $N$ a $r-k$ prvků z $M$,
          neboli:
          \[
            \sum_{k=0}^{n} \binom{n}{k}\binom{m}{r-k}
          \]
          Nevadí, když bude $r-k$ záporné, protože se dané prvky vynulují.
        \end{enumerate}
        
        Když se dá výsledek tohoto příkladu vyjádřit dvěma způsoby, pak oba výrazy jsou si rovny:
        \[
          \sum_{k=0}^{n} \binom{n}{k}\binom{m}{r-k} = \binom{n+m}{r}
        \]
        Q.E.D.

      \item \textbf{Důkaz 2} - indukce podle $n$:
        \begin{enumerate}
          \item $n=0$:
            \begin{align*}
              \binom{0}{0}\binom{m}{r} &= \binom{0+m}{r} \\
              \binom{m}{r} &= \binom{m}{r} \\
            \end{align*}
          \item $n+1$:

            Indukční předpoklad (IP):
            \[
              \sum_{k=0}^{n} \binom{n}{k}\binom{m}{r-k} = \binom{n+m}{r}
            \]

            Chceme:
            \[
              \sum_{k=0}^{n+1} \binom{n+1}{k}\binom{m}{r-k} = \binom{n+1+m}{r}
            \]

            Ekvivalentními úpravami dokažme vzorec pro $n+1$:
            \begin{align*}
              \sum_{k=0}^{n+1} \left(\binom{n}{k-1}+\binom{n}{k}\right)\binom{m}{r-k} = \binom{n+1+m}{r} \\
              \overset{\text{Pascal}}{\Longleftrightarrow}
              \sum_{k=0}^{n+1} \binom{n}{k-1}\binom{m}{r-k} + \sum_{k=0}^{n+1} \binom{n}{k}\binom{m}{r-k} = \binom{n+1+m}{r}
            \end{align*}
            {\scriptsize \begin{align*}
              \text{Upravme oba členy levé strany} \\
              \sum_{k=0}^{n+1} \binom{n}{k-1}\binom{m}{r-k} 
              &= \underbrace{\binom{n}{-1}\binom{m}{r}}_{0} + \sum_{k=1}^{n+1} \binom{n}{k-1}\binom{m}{r-k}
              \overset{\text{j+1:=k}}{=} \sum_{j=0}^{n} \binom{n}{(j+1)-1}\binom{m}{r-(j+1)} 
              = \sum_{j=0}^{n} \binom{n}{j}\binom{m}{(r-1)-j}
              \\
              \sum_{k=0}^{n+1} \binom{n}{k}\binom{m}{r-k}
              &= \sum_{k=0}^{n} \binom{n}{k}\binom{m}{r-k} + \underbrace{\binom{n}{n+1}\binom{m}{r-k}}_{0}
              = \sum_{k=0}^{n} \binom{n}{k}\binom{m}{r-k}
            \end{align*}}
            \begin{align*}
              &\sum_{k=0}^{n} \binom{n}{k}\binom{m}{(r-1)-k} + \sum_{k=0}^{n} \binom{n}{k}\binom{m}{r-k} = \binom{n+1+m}{r} \\
              \overset{IP}{\Longleftrightarrow}
              &\sum_{k=0}^{n} \binom{n}{k}\binom{m}{(r-1)-k} + \binom{n+m}{r} = \binom{n+1+m}{r} \\
              &\text{Když IP platí pro libovolné $r$, pak platí i pro $r-1$} \\
              \overset{\text{IP pro r-1}}{\Longleftrightarrow}
              &\binom{n+m}{r-1} + \binom{n+m}{r} = \binom{n+1+m}{r} \\
              \overset{\text{Pascal}}{\Longleftrightarrow}
              &\binom{n+m+1}{r} = \binom{n+1+m}{r}
            \end{align*}
            Q.E.D
        \end{enumerate}
    \end{enumerate}
\end{enumerate}



\end{document}
