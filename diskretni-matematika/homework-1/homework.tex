\documentclass[a4paper,12pt]{article}
\usepackage{amsmath}
\usepackage{xcolor}
\usepackage{amssymb}


\begin{document}

\section*{Domácí úkol I}

Vypracoval: Daniel \textit{"Randál"} Ransdorf \hfill Podpis: \rule{4cm}{0.4pt}

\begin{enumerate}
  \item \textbf{Vlastnosti součtu:}
    \[
      S_n = \sum_{k=1}^n k = \frac{n(n+1)}{2}.
    \]

    \begin{enumerate}
      \item \textbf{Teze:} 
        Když je $S_n$ \colorbox{yellow}{liché}, pak $n \equiv 1 \lor n \equiv 2 \pmod{4}$.

      \item \textbf{Důkaz:}
        \[
        S_n = \sum_{k=1}^{n} k
        \]
        \[
          S_n = 1+2+3+4+5+...+n
        \]
        \[
          S_n \equiv \underbrace{1+0+1+0+1+...+n}_{\text{počet jedniček }=\lceil \frac{n}{2}\rceil} \pmod{2}
        \]
        \[
          S_n \equiv \lceil \frac{n}{2} \rceil \pmod{2}
        \]

        Nyní rozdělíme podle zbytku \(n \pmod{4}\):

        \[
        \begin{aligned}
        n &= 4m:   &\quad \left\lceil \tfrac{n}{2} \right\rceil = 2m \equiv 0 \pmod{2} &\;\;\Rightarrow S_n \equiv 0 \pmod{2}, \\
        n &= 4m+ \colorbox{green}{1}: &\quad \left\lceil \tfrac{n}{2} \right\rceil = 2m+1 \equiv 1 \pmod{2} &\;\;\Rightarrow S_n \equiv \colorbox{yellow}{$1$} \pmod{2}, \\
        n &= 4m+ \colorbox{green}{2}: &\quad \left\lceil \tfrac{n}{2} \right\rceil = 2m+1 \equiv 1 \pmod{2} &\;\;\Rightarrow S_n \equiv \colorbox{yellow}{$1$} \pmod{2}, \\
        n &= 4m+3: &\quad \left\lceil \tfrac{n}{2} \right\rceil = 2m+2 \equiv 0 \pmod{2} &\;\;\Rightarrow S_n \equiv 0 \pmod{2}.
        \end{aligned}
        \]

        Když je $n \equiv 0  \text{ nebo }  3 \pmod{4}$, pak je \(S_n\) sudé.
        Z toho plyne, že když je \(S_n\) liché, pak $n \equiv \colorbox{green}{1} \ \text{nebo} \ \colorbox{green}{2} \pmod{4}$. $\Box$
    \end{enumerate}
  
  \pagebreak

  \item \textbf{Trojúhelníková nerovnost:} 
    
    \begin{enumerate}
      \item \textbf{Teze:} 
        Pro všechna reálná $a,b$ platí:
        \[
          |a+b| \le |a| + |b|
        \]
      
      \item \textbf{Důkaz:} \\
        Vycházejíce z vlastnosti ...
          \[\forall x,y \ge 0:\quad \bigl(x \le y \;\Longleftrightarrow\; x^{2} \le y^{2}\bigr)\]
        ... umocněme nerovnici na druhou:
          \begin{align*}
          (|a+b|)^2 &\le (|a| + |b|)^2 \quad &\text{(roznásobení)} \\[1ex]
          a^2 + 2ab + b^2 &\le a^2 + 2|a||b| + b^2 \quad &\text{($-a^2 - b^2$)} \\[1ex]
          2ab &\le 2|a||b| \quad &\text{($\frac{...}{2}$)} \\[1ex]
          ab &\le |a||b| \\[1ex]
          ab &\le |ab|
          \end{align*}

          A to platí vždy, neboť $x \le |x|$ pro všechna $x\in\mathbb{R}$. Tím je nerovnost dokázána. $\Box$
      
          \item \textbf{Pozorování rovnosti:} \\
            V každém kroku výše šlo o ekvivalence, proto ...
            \[
            |a+b| = |a|+|b| \;\Longleftrightarrow\; ab = |ab| \;\Longleftrightarrow\; ab\ge 0.
            \]

            ... z toho plyne, že $|a+b| = |a|+|b|$ platí pouze, když $ab \geq 0$
    \end{enumerate}
\end{enumerate}



\end{document}
