\documentclass[a4paper,12pt]{article}
\usepackage[czech]{babel}
\usepackage{amsmath,amsthm,amssymb}
\usepackage{tikz}
\usetikzlibrary{calc}

\title{Grafický důkaz: dlažba L--trominy na \texorpdfstring{$2^n\times 2^n$}{2^n x 2^n} šachovnici s chybějícím rohem}
\author{}
\date{}

\newtheorem*{thm}{Tvrzení}
\newtheorem*{proofsketch}{Náčrt důkazu}

% Pomůcka: vykreslení mřížky N x N se zvýrazněním vybraných buněk
% souřadnice buněk bereme jako dolní-levé rohy (i,j), i,j \in \{0,\dots,N-1\}
\newcommand{\DrawGrid}[3]{%
  % #1 = N (rozměr), #2 = tloušťka čáry, #3 = měřítko (cm na jednotku)
  \begin{tikzpicture}[x=#3cm,y=#3cm]
    % mříž
    \foreach \i in {0,...,#1} {
      \draw[line width=#2] (\i,0) -- (\i,#1);
      \draw[line width=#2] (0,\i) -- (#1,\i);
    }
  \end{tikzpicture}%
}

% Obrázek: 2x2 báze s jedním chybějícím rohem (levý horní)
\newcommand{\BaseCaseTwoByTwo}{%
\begin{tikzpicture}[x=0.8cm,y=0.8cm]
  % mříž 2x2
  \foreach \i in {0,...,2} {
    \draw (\i,0)--(\i,2);
    \draw (0,\i)--(2,\i);
  }
  % chybějící levý horní roh (0,1)-(1,2)
  \fill[black!20] (0,1) rectangle (1,2);
  \node at (0.5,1.5) {\footnotesize \textbf{$\times$}};
  % L-tromino zbylých tří buněk
  \fill[blue!35] (1,0) rectangle (2,1);
  \fill[blue!35] (0,0) rectangle (1,1);
  \fill[blue!35] (1,1) rectangle (2,2);
  % pomocné popisky
  \node[anchor=north] at (1,0) {\scriptsize};
\end{tikzpicture}%
}

\newcommand{\InductiveStepFourByFour}{%
\begin{tikzpicture}[x=0.35cm,y=0.35cm]
   % celá mříž 8x8
  \foreach \i in {0,...,4} {
    \draw (\i,0)--(\i,4);
    \draw (0,\i)--(4,\i);
  }
  \fill[white!20] (-0.1,3.1) rectangle (0.9,4.1);

\end{tikzpicture}%
}

\newcommand{\InductiveStepEightByEight}{%
\begin{tikzpicture}[x=0.35cm,y=0.35cm]
  % celá mříž 8x8
  \foreach \i in {0,...,8} {
    \draw (\i,0)--(\i,8);
    \draw (0,\i)--(8,\i);
  }
  % chybějící levý horní roh: buňka (0,7)-(1,8)
  \fill[white!20] (-0.1,7.1) rectangle (0.9,8.1);
  % dělící čáry (střed) pro rozklad na 4 kvadranty 4x4
  \draw[line width=1.2pt,red] (4,0)--(4,8);
  \draw[line width=1.2pt,red] (0,4)--(8,4);
  % centrální 2x2 blok: souřadnice LL rohů jsou (3,3), (4,3), (3,4), (4,4)
  % protože chybějící roh je v severozápadním kvadrantu, NEZASAHUJEME do čtverce (3,4)
  % a položíme jedno L-tromino přes tři zbývající středové buňky:
  \fill[blue!35] (3,3) rectangle (4,4);
  \fill[blue!35] (4,3) rectangle (5,4);
  \fill[blue!35] (4,4) rectangle (5,5);
  % naznačení rekurze: vybledlé šrafování pro "chybějící rohy" v ostatních kvadrantech
  
  % popisek středu
\end{tikzpicture}%
}

\newcommand{\InductiveStepEightByEightSeparated}{%
\begin{tikzpicture}[x=0.35cm,y=0.35cm]
  % celá mříž 8x8
  \foreach \i in {0,...,4} {
    \draw (\i,0)--(\i,4);
    \draw (0,\i)--(4,\i);
  }
  \foreach \i in {0,...,4} {
    \draw (\i,4.5)--(\i,8.5);
    \draw (4.5,\i)--(8.5,\i);
  }
  \foreach \i in {4.5,...,8.5} {
    \draw (\i,4.5)--(\i,8.5);
    \draw (4.5,\i)--(8.5,\i);
  }
  \foreach \i in {4.5,...,8.5} {
    \draw (\i,0)--(\i,4);
    \draw (0,\i)--(4,\i);
  }
  % chybějící levý horní roh: buňka (0,7)-(1,8)
  \fill[white!20] (-0.1,7.6) rectangle (0.9,8.6);
  \fill[white!20] (4.4,4.4) rectangle (5.4,5.4);
  \fill[white!20] (4.4,3.1) rectangle (5.4,4.1);
  \fill[white!20] (3.1,3.1) rectangle (4.1,4.1);
\end{tikzpicture}%
}

\begin{document}


\section*{Domácí úkol I}

Vypracoval: Daniel \textit{"Randál"} Ransdorf \hfill Podpis: \rule{4cm}{0.4pt}

\begin{enumerate}
  \item \textbf{Šachovnice:}

    Šachovnici $2^1 \times 2^1$ složím, protože se z ní odebráním rohů stane mnohoúhelník ve tvaru L ze zadání.
    S předpokladem, že umíme složit šachovnici $2^n \times 2^n$ můžeme indukovat, že složíme $2^{n+1} \times 2^{n+1}$.
    Uděláme to tak, že šachovnici $2^{n+1} \times 2^{n+1}$ rozdělíme na čtyři šachovnice $2^n \times 2^n$.
    Veprostřed odebereme L dílek tak, aby sebral jeden čtvereček třem šachovnicím bez odebraného rohu.
    Vzniknou nám 4 šachovnice $2^n \times 2^n$, které víme, že umíme složit.
    \usetikzlibrary{patterns}
    \[
    \InductiveStepFourByFour
    \;\Rightarrow\;
    \InductiveStepEightByEight
    \;\Rightarrow\;
    \InductiveStepEightByEightSeparated
    \]

    Proto lze pokrýt i $2^{n+1}\times 2^{n+1}$. Indukcí podle $n$ je tvrzení dokázáno.

  \item \textbf{Suma:}

    \[
      \sum_{i=1}^{n}i^3 = \left( \sum_{i=1}^{n}i \right)^2
    \]

    $n=0$ splňuje rovnost:

    \[
    0=0
    \]
    $n=1$ splňuje rovnost:

    \[
    1^3 = (1)^2
    \]

    Chceme dokázat, že když rovnost platí pro $n$, pak i pro $n+1$:
    
    \begin{align*}
      \text{IP: } \sum_{i=1}^{n}i^3 &= \left( \sum_{i=1}^{n}i \right)^2 \\
      1^3+2^3+...+n^3 &= (1+2+...+n)^2 \\
      1^3+2^3+...+n^3 + (n+1)^3 &= (1+2+...+n)^2 + (n+1)^3 \\
      \sum_{i=1}^{n+1}i^3 &= \left( \frac{n(n+1)}{2} \right)^2 + (n+1)^3 \\
      \sum_{i=1}^{n+1}i^3 &= \frac{n^2(n+1)^2}{4} + (n+1)^3 \\
      \sum_{i=1}^{n+1}i^3 &= (n+1)^2 \left(\frac{n^2}{4} + (n+1) \right) \\
      \sum_{i=1}^{n+1}i^3 &= \frac{(n+1)^2(n^2+4n+4)}{4}\\
      \sum_{i=1}^{n+1}i^3 &= \frac{(n+1)^2(n+2)^2}{4}\\
      \sum_{i=1}^{n+1}i^3 &= \left( \frac{(n+1)(n+2)}{2} \right)^2\\
      \sum_{i=1}^{n+1}i^3 &= \left( \sum_{i=1}^{n+1}i \right)^2\\
    \end{align*}

    Rovnost platí pro $n+1$, tím pádem pro všechny $n \in \mathbb{N}$

\end{enumerate}



\end{document}
