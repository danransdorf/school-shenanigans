\documentclass[12pt]{article}
\usepackage[a3paper, margin=0.4in]{geometry}
\usepackage{amsmath}
\usepackage{xcolor}
\usepackage{amssymb}


\begin{document}

\section*{Domácí úkol IV}

Vypracoval: Daniel \textit{"Randál"} Ransdorf \hfill Podpis: \rule{4cm}{0.4pt}

\begin{enumerate}
  \item \textbf{Různé cesty}
    Musíme dojít z levého dolního rohu do pravého horního, tedy musíme udělat
    $m$ kroků nahoru a $n$ kroků doprava. Dohromady uděláme $m+n$ kroků.
    Musíme tedy vypočítat, kolika způsoby můžeme do sekvence $m+n$ kroků
    zařadit $n$ kroků doprava, nebo $m$ kroků nahoru (výsledek bude stejný).
    Máme tedy vybrat $m$/$n$ míst z $m+n$, tedy:

    \[ \binom{m+n}{m} \quad \text{nebo} \quad \binom{m+n}{n}\]

    Dvě varianty jsou si rovny kvůli rovnosti, kterou známe z přednášky:

    \[ \binom{n}{k} = \binom{n}{n-k} \]
  
  \item \textbf{Slova}
    Písmen od A do P je 16 (CH a háčky nepočítáme). 
    Celkových možností, jak seřadit čísla je zjevně $16!$.
    Odečteme od toho počet možných seřazení PONK, DOBA, COP.
    Nazpět příčteme průniky seřazení PONK, DOBA, COP.

    \begin{itemize}
      \item \textbf{PONK}: Předpokládejme, že písmena jsou takto za sebou,
        počítáme kolika způsoby mohou v sekvenci být?
        Vybíráme tedy 4 místa z 16 a pak započítám seřazení zbylých 12 čísel ($12!$), takže
        \[ \binom{16}{4} \cdot 12! = \frac{16!}{4!} \]
      \item \textbf{DOBA}: Stejný případ jako PONK, máme 4 písmena seřazena
        \[ \binom{16}{4} \cdot 12! = \frac{16!}{4!} \]
      \item \textbf{COP}: Pro cop vybíráme 3 místa z 16 a seřadíme zbylých 13 písmen, takže
        \[ \binom{16}{3} \cdot 13! = \frac{16!}{3!} \]
      \item \textbf{PONK $\cap$ DOBA}: Předpokládejme, že 7 unikátních písmen
        z těchto dvou slov jsou uspořádané tak, že jsou obě slova splněny.
        Před O můžeme P a D uspořádat dvěma způsoby, za O můžeme N,K,B,A uspořádat
        NKBA, NBKA, NBAK, BNAK, BANK, čili šesti způsoby.
        Protože jsou tyto dva podproblemy na sobě zjevně nezávislé,
        počet možností, jak můžeme seřadit PONKDBA je $6 \cdot 2 = 12$.
        Pro těchto sedm písmen můžeme stejným způsobem, jako u minulých případů,
        vybrat 7 míst z 16 a zbylých 9 písmen libovolně seřadit.
        \[ 12 \cdot \binom{16}{7} \cdot 9! = \frac{12\cdot16!}{7!} \]
      \item \textbf{DOBA $\cap$ COP}: Předpokládejme, že 6 unikátních písmen
        z těchto dvou slov jsou uspořádané tak, že jsou obě slova splněny.
        Před O můžeme C a D uspořádat dvěma způsoby, za O můžeme P,B,A uspořádat
        PBA, BPA, BAP, čili třemi způsoby.
        Protože jsou tyto dva podproblemy na sobě zjevně nezávislé,
        počet možností, jak můžeme seřadit DOBACP je $2 \cdot 3 = 6$.
        Pro těchto šest písmen můžeme stejným způsobem, jako u minulých případů,
        vybrat 6 míst z 16 a zbylých 10 písmen libovolně seřadit.
        \[ 6 \cdot \binom{16}{6} \cdot 10! = \frac{6\cdot16!}{6!} = \frac{16!}{5!} \]
      \item \textbf{PONK $\cap$ COP}: Tyto dvě slova mají prázdný průnik,
        protože jedno má P před O a druhé O před P.
    \end{itemize}

    Podle původní úvahy můžeme postavit jednoduchý "pseudovzorec" pro výpočet chtěných pořadí.

    \begin{align*}
      \text{\# chtěných pořadí} = \text{\# celkových pořadí} -
      (|PONK| + |DOBA| + |COP| - (|PONK \cap DOBA| + |DOBA \cap COP| + |PONK \cap COP|)) \\
      \Leftrightarrow \text{\# chtěných pořadí} = \text{\# celkových pořadí} -
      |PONK| - |DOBA| - |COP| + |PONK \cap DOBA| + |DOBA \cap COP| + |PONK \cap COP|
    \end{align*}

    Dosazením dostaneme

    \begin{align*}
      &6! - \frac{16!}{4!} - \frac{16!}{4!} - \frac{16!}{3!} + \frac{12\cdot16!}{7!} + \frac{16!}{5!} + 0 \\
      &= 16! (1 - \frac{1}{4!} - \frac{1}{4!} - \frac{1}{3!} + \frac{12}{7!} + \frac{1}{5!}) \\
      &= \underline{\underline{15916265164800}}
    \end{align*}
\end{enumerate}



\end{document}
