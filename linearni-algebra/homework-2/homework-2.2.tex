\documentclass[12pt]{article}
\usepackage[a4paper, margin=1in]{geometry}
\usepackage{amsmath, amssymb, amsthm, ulem}

\begin{document}

\section*{Domácí úkol I}

Vypracoval: Daniel \textit{"Randál"} Ransdorf \hfill Podpis: \rule{4cm}{0.4pt}

\begin{enumerate}
  \setcounter{enumi}{1}
  \item \textbf{Řešení}:
    \[
    p = \left\{ 
      \begin{pmatrix} 1 \\ 2 \\ 4 \end{pmatrix} 
      + t \begin{pmatrix} 1 \\ 0 \\ 1 \end{pmatrix} 
      \;:\; t \in \mathbb{R}
    \right\}
    \]
    \begin{enumerate}
      \item Pro která $a,b,c,d \in \mathbb{R}$ obsahuje rovina s obecnou rovnicí $ax+by+cz=d$ přímku $p$?

        Taková rovina musí splňovat dvě kritéria: musí být rovnoběžná s přímkou, musí v rovině ležet alespoň jeden bod přímky.
        
        Rovnoběžnost vyšetříme porovnáním normálového vektoru roviny a směrového vektoru přímky.
        Tyto vektory na sebe musí být kolmé. Zda v rovině leží alespoň jeden bod přímky
        zjistíme dosazením pevného bodu $(1,2,4)$ do obecné rovnice roviny.

        \begin{align*}
          \begin{pmatrix}
            a \\ b \\ c
          \end{pmatrix} \cdot
          \begin{pmatrix}
            1 \\ 0 \\ 1
          \end{pmatrix}
          = 0 \\
          a+c = 0 \\
          c = -a
      \end{align*}
      \begin{align*}
        a\cdot1 + b\cdot2 + c\cdot4 = d \\
        a\cdot1 + b\cdot2 + (-a)\cdot4 = d \\
        2b - 3a = d \\
      \end{align*}
      Pomocí proměnných $a,b$ dokážeme vyjádřit $c,d$. $a,b,c$ nesmí být 0, protože by pak útvar nebyl rovina.
      Zvolme tedy parametry $t_1,t_2 \in \mathbb{R} \setminus \{0\}$ a dosaďme je za $a,b$:
      \begin{align*}
        a&=t_1 \\
        b&=t_2 \\
        c&=-a=-t_1 \\
        d&=2b-3a=2t_2-3t_1 \\
        \begin{pmatrix}
          a \\ b \\ c \\ d
        \end{pmatrix} &=
        \begin{pmatrix}
          t_1 \\ t_2 \\ -t_1 \\ 2t_2-3t_1
        \end{pmatrix} =
        \begin{pmatrix}
          t_1 \\ 0 \\ -t_1 \\ -3t_1
        \end{pmatrix} +
        \begin{pmatrix}
          0 \\ t_2 \\ 0 \\ 2t_2
        \end{pmatrix} = t_1
        \begin{pmatrix}
          1 \\ 0 \\ -1 \\ -3
        \end{pmatrix} +
        t_2
        \begin{pmatrix}
          0 \\ 1 \\ 0 \\ 2
        \end{pmatrix}
      \end{align*}
      \[
        \underline{\underline{
          \begin{pmatrix}
            a \\ b \\ c \\ d
          \end{pmatrix} \in \left\{
            t_1
            \begin{pmatrix}
              1 \\ 0 \\ -1 \\ -3
            \end{pmatrix} +
            t_2
            \begin{pmatrix}
              0 \\ 1 \\ 0 \\ 2
            \end{pmatrix} 
            \quad \middle| \quad t_1,t_2 \in \mathbb{R} \setminus \{0\} 
            \right\} 
          }}
      \]

      \item Dosazením $t_1,t_2=1$ získáme první rovnici roviny, která obsahuje přímku. Dosazením $t_1=1,t_2=2$ dostaneme druhou.
            \[
            t_1,t_2=1 \;\Rightarrow\; \begin{pmatrix}
              a\\b\\c\\d
            \end{pmatrix} = \begin{pmatrix}
              1 \\ 1 \\ -1 \\ -1
            \end{pmatrix} \;\Rightarrow\; x+y-z=-1
            \]
            \[
            t_1=1,t_2=2 \;\Rightarrow\; \begin{pmatrix}
              a\\b\\c\\d
            \end{pmatrix} = \begin{pmatrix}
              1 \\ 2\\-1\\1
            \end{pmatrix} \;\Rightarrow\; x+2y-z=1
            \]

            Našli jsme soustavu dvou lineárních rovnic o třech neznámých:

            \begin{align*}
              x+y-z &= -1\\
              x+2y-z &= 1
            \end{align*}
    \end{enumerate}
\end{enumerate}

\end{document}
