\documentclass[12pt]{article}
\usepackage[a4paper, margin=1in]{geometry}
\usepackage{amsmath, amssymb, amsthm, ulem}

\begin{document}

\section*{Domácí úkol I}

Vypracoval: Daniel \textit{"Randál"} Ransdorf \hfill Podpis: \rule{4cm}{0.4pt}

\begin{enumerate}
  \item \textbf{Řešení}:
    Zobrazení bude dle definice "na", když:
    \[
      \forall (c,d) \in \mathbb{R}^2 \exists x,y \quad : \quad f_a(x,y) = (c,d)
    \]
    neboli soustava rovnic níže musí mít řešení pro libovolné $c,d \in \mathbb{R}$
    
    \[
    \begin{cases}
    (a+2)x + ay = c,\\
    ax + y = d.
    \end{cases}
    \]

    Z druhé rovnice: $y = d - ax$
    \begin{align*}
    \Rightarrow (a+2)x + a(d - ax) &= c \\
    (a+2)x + ad - a^2x &= c \\
    x(a+2 - a^2) &= c - ad
    \end{align*}
    \[
    x = \dfrac{c - ad}{a+2 - a^2} = \dfrac{ad - c}{a^2 - a - 2}
    \]

    \[
    y = d - a\left(\dfrac{ad - c}{a^2 - a - 2}\right)
    \]
    Podmínka pro existenci řešení pro všechny $c,d \in \mathbb{R}$:
    \begin{align*}
    a^2 - a - 2 &\ne 0 \\
    \Rightarrow (a-2)(a+1) &\ne 0 \\
    \Rightarrow a \ne 2 \land a \ne -1
    \end{align*}

    Řešení pro libovolné $c,d \in \mathbb{R}$ existuje pro $a \in \mathbb{R} \setminus \{-1,2\}$.
    Z toho plyne, že \underline{$f_a$ je "na", právě když $a \in \mathbb{R} \setminus \{-1,2\}$.}
    
    Ověření vyloučených možností:
    \[
      a=-1: \quad
      \begin{cases}
      x - y = c,\\
      -x + y = d.
      \end{cases}
      \Rightarrow c + d = 0
    \]
    \[
      a=2: \quad
      \begin{cases}
      4x + 2y = c,\\
      2x + y = d.
      \end{cases}
      \Rightarrow c - 2d = 0
    \]

    U obou možností existuje $c,d$ takové, že soustava nemá řešení (např. $c=0,d=1$). Tedy existuje $c,d$ takové, že pro všechna $x,y$ platí $f_a(x,y) \ne (c,d)$.
\end{enumerate}

\end{document}
