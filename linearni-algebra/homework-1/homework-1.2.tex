\documentclass[12pt]{article}
\usepackage[a4paper, margin=1in]{geometry}
\usepackage{amsmath, amssymb, amsthm, ulem}

\begin{document}

\section*{Domácí úkol I}

Vypracoval: Daniel \textit{"Randál"} Ransdorf \hfill Podpis: \rule{4cm}{0.4pt}

\begin{enumerate}
  \setcounter{enumi}{1}

  \item \textbf{Řešení}:
    
    Princip: Osová souměrnost je zobrazení, které bodu $A$ přiřazuje obraz $A'$ takový,
    že zadaná přímka $o$ prochází středem úsečky $AA'$ a úsečka $AA' \perp o$.
    Z toho plyne, že zvolíme-li vektor $\mathbf{v}$ kolmý na $o$ takový, že $A+\mathbf{v} \in o$, pak $A' = A+2\mathbf{v}$.
    Vektor $\mathbf{v}$ je kolmý na $o$, takže se musí dát vyjádřit jako $k$-násobek normálového vektoru $\mathbf{n}_o$ přímky $o$.
    Z toho plyne:
    \[
      A' = A + 2k\mathbf{n}_o \quad , A + k\mathbf{n}_o \in o
    \]
    
    Pro jednoduchost nejprve najděme obecnou rovnici přímky $p$ a z obecných rovnic najděme normálové vektory přímek.
    \begin{align*}
      p &= \{(1, 0) + t(0, 1) : t \in \mathbb{R}\} \\
      (x, y) &= (1, 0) + t(0, 1) \\
      x &= 1, \quad y = t \\
      \Rightarrow\; x - 1 &= 0 \quad \text{(obecná rovnice přímky $p$)}
    \end{align*}
    \[
      \mathbf{n}_p = (1, 0), \quad \mathbf{n}_q = (1, 1) \quad \text{(normálové vektory přímek)}
    \]
    \begin{enumerate}
      \item
      \begin{align*}
        O_p: (x,y) &\mapsto (x,y) + 2k\mathbf{n}_p \quad , (x,y) + k\mathbf{n}_p \in p \\
        O_p: (x,y) &\mapsto (x,y) + 2k\mathbf{n}_p \quad , (x,y) + (k, 0) \in p \\
        O_p: (x,y) &\mapsto (x,y) + 2k(1,0) \quad , (x+k,y) \in p \\
        O_p: (x,y) &\mapsto (x,y) + 2k(1,0) \quad , (x+k)-1=0 \quad \Leftrightarrow k=1-x \\
        O_p: (x,y) &\mapsto (x,y) + 2(1-x)(1,0) \\
        O_p: (x,y) &\mapsto (x+2-2x,y) \\
        O_p: (x,y) &\mapsto \underline{\underline{(-x+2,y)}}
      \end{align*}
      \begin{align*}
        O_q: (x,y) &\mapsto (x,y) + 2k\mathbf{n}_q \quad , (x,y) + k\mathbf{n}_q \in q \\
        O_q: (x,y) &\mapsto (x,y) + 2k\mathbf{n}_q \quad , (x,y) + (k,k) \in q \\
        O_q: (x,y) &\mapsto (x,y) + 2k(1,1) \quad , (x+k,y+k) \in q \\
        O_q: (x,y) &\mapsto (x,y) + 2k(1,1) \quad , (x+k)+(y+k)=4 \quad \Leftrightarrow 2k=4-x-y \\
        O_q: (x,y) &\mapsto (x,y) + (4-x-y)(1,1) \\
        O_q: (x,y) &\mapsto (x+4-x-y, y+4-x-y) \\
        O_q: (x,y) &\mapsto \underline{\underline{(4-y, 4-x)}}
      \end{align*}

      \item
      \begin{align*}
        O_p \circ O_q: (x,y) &\mapsto O_p(O_q(x,y)) \\
        O_p \circ O_q: (x,y) &\mapsto O_p(a,b) \quad , (a,b) = O_q(x,y) \\
        O_p \circ O_q: (x,y) &\mapsto (-a+2, b) \quad , (a,b) = (4-y, 4-x) \\
        O_p \circ O_q: (x,y) &\mapsto (-(4-y)+2, 4-x) \quad \\
        O_p \circ O_q: (x,y) &\mapsto \underline{\underline{(y-2, 4-x)}}
      \end{align*}
      \begin{align*}
        O_q \circ O_p: (x,y) &\mapsto O_q(O_p(x,y)) \\
        O_q \circ O_p: (x,y) &\mapsto O_q(a,b) \quad , (a,b) = O_p(x,y) \\
        O_q \circ O_p: (x,y) &\mapsto (4-b, 4-a) \quad , (a,b) = (-x+2, y) \\
        O_q \circ O_p: (x,y) &\mapsto (4-y, 4-(-x+2)) \\
        O_q \circ O_p: (x,y) &\mapsto \underline{\underline{(4-y, x+2)}}
      \end{align*}
    \end{enumerate}
\end{enumerate}

\end{document}
